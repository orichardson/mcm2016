% !TeX root = intro.tex
\documentclass[paper.tex]{subfiles}

\begin{document}
	\section{Introduction}
	Academic success among undergraduate students at universities in the U.S. depends on several factors such as teacher capability, class size, and university funding. In nearly every statistical study regarding factors that contribute to student success, the amount of funding per student at each school always held the greatest positive impact.\cite{greenwald1996effect}  While great teachers and small class sizes certainly effect student success in the classroom, we know that adequate funding is essential to resource expansion in U.S. schools.\cite{hedges1994exchange} Additional resources that are readily available to the student have been shown to increase student performance across the board. 

	If a foundation should want to increase student success by providing funds to several universities, the foundation could pursue one of several options.
	
	Most grant-awarding foundations that are prominent today allocate funds according to a recipients's qualifications. These qualifications are usually determined via a selection process that involves an application or proposal from the university, followed by a direct review of that university by the foundation. Successful candidates for grants usually demonstrate adequate need for the award, as well as a plan to use the money to ameliorate the situation at the school. Universities with both need and potential for growth are selected for grants primarily by human opinion and group decision. (unsure how to cite this paragraph, information came from browsing various foundation websites)

	In order to avoid this somewhat meticulous selection process, we develop a model that will rank and allot appropriate funds to each university for the foundation. The model employs methods of data clustering, neural networks, and multi-objective programming to essentially replace the human-performed decision making process of previously used grant allocation techniques.
	
	We propose a model that will rank universities according to their eligibility for the grant, amount of grant money to be received, and rate at which the grant money will be distributed.
	 
	\begin{itemize} 
		\item[$\hookrightarrow$] The model ranks all universities according to their current available funds due to outside donations each year in addition to how each university employs those funds. Universities with relatively small donation pools and a loyal history of fund allocation to expansion of student resources receive high rankings.
		\item[$\hookrightarrow$] The model approximates the amount of effective change it can induce by giving funds to schools of various rankings. The schools with optimized rates of changes will receive the largest grants from the Foundation.
		\item[$\hookrightarrow$] The size of grant awarded to each school in the list will be determined by XXXXXXX.
		\item[$\hookrightarrow$] Over a period of five years, the foundation will award each grant according to a predictive distribution of optimal funding per year at each institution. This data comes from published financial data regarding university donation spending.
	\end{itemize}
	
	
	\section{Assumptions}
Utility Assumptions:
	\begin{enumerate}
		\item The utility function will fall off toward a constant value as grant size goes to infinity.
		\item The Foundation will not allocate "negative grants," that is, the utility function will only contain positive values.
		\item The utility function will be equal or nearly equal to zero when the grant award is also equal to zero. This prevents universities with a score of zero from receiving undeserved funds.
	\end{enumerate}
\end{document}