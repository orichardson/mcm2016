\documentclass[paper.tex]{subfiles}

\begin{document}
	\section*{Appendix A: Score Derivation}
	If we let $\xi_i$ be the $i^{th}$ component of $V_u + \int_{T} F(w_T, u)~dt$, then we have
	\begin{align*}
		S(u) &=\frac{d}{dM_u} A\left( \xi \right) \\
		&= \sum_i \frac{\partial A}{\partial v_i} \frac{d \xi_i}{d M_u}\\
		&= \sum_i \frac{\partial A}{\partial v_i} \frac{d}{d M_u} \left(v_{i,u} + \int_{T} f_i(w_T, u)~dt\right) \\
		&= \sum_i \frac{\partial A}{\partial v_i}  \left( \int_{T}\frac{\partial }{\partial M_u} f_{i,u}(w_T)~dt\right) \\
	\intertext{\ldots by Leibeitz' Rule, since the time frame limits of integration do not not depend on $M_u$. Because of the associativity of maps, we can write the composition of $F$ with $w_T$ as a function of $t$: }
		S(t)&= \sum_i \frac{\partial A}{\partial v_i}  \left( \int_{T}\frac{\partial }{\partial M_u} (f_{i,u} \circ w_T)(t)~dt\right) \\
		%&= \sum_i \frac{\partial A}{\partial v_i}  \left(~~\iint\limits_{t_1, t_2 \in T} \frac{\partial f_{i,u}}{\partial w_{T(t_2)}(t_1)} \frac{d w_{T(t_2)} (t_1)}{d M_u} ~dt_1\right)
	\end{align*}
	
	Because of $T$ is discrete, $w_T$ is really just an $n$-vector over $\D$, and so our score function becomes
	
	\begin{align*}
		S(u) &= \sum_i \frac{\partial A}{\partial v_i} \sum_{t_j} \frac{\partial }{\partial M_u} f_{i,u}\Big(D_{t_j}, D_{t_{j-1}}, \ldots, D_{t_{j-n+1}} \Big)\\
 		 &= \sum_i \frac{\partial A}{\partial v_i} \sum_{t_j} \sum_m \boxed{\frac{\partial f_{i,u}}{\partial x_m}}~\frac{d x_m }{d M_u}\Big(\{x_m\} \Big)
	\end{align*}
	\ldots where $\{x_m\}$ is the expanded representation of the $D_i$ variables, and $x_m$ is the $m^{th}$ real argument to $f$.
	This is still pretty ugly, but fortunately, we can make sense of these quantities. The total amount of money given is the sum of the specific $x_m$ variables that align with basis elements of $\mathcal{M}$.
	\[ M = \sum_{x_l \in \mathscr{B}\mathcal{M}} x_l \]
	\[ \Rightarrow \qquad \frac{dx_m}{dM} = \begin{cases}1 - \sum\limits_{m \neq l, x_l \in \mathscr{B}\mathcal{M}} x_l & \text{if $x_m \in \mathscr{B}\mathcal{M}$} \\ 0 & \text{otherwise}\end{cases} \]
	Fortunately, we're summing over all of these elements, and so we have
	
	The boxed quantity is a number we can recover from the weights of our trained function $f$. The number to its right is even simpler; $x_m$ is either $M_u$ or an independent variable, and so if it simplifies to $\delta_{mk}$
	
	In the case that $A$ is just the linear function $A(V) = \sum_i a_i v_i$, we can write the score completely in terms of known variables:
	
	\begin{equation}
		S(u) = \sum_i a_i \sum_{t_j, m} \psi 
	\end{equation}
	
\end{document}
