\documentclass[11pt, reqno]{amsart}

\edef\restoreparindent{\parindent=\the\parindent\relax}
\usepackage{parskip}
\restoreparindent
\usepackage{bbm}

\usepackage{enumitem}
\usepackage{amsmath, amssymb, mathrsfs}
\usepackage{graphicx}
\usepackage{url}
\usepackage{wallpaper}
\usepackage{subfiles}
\usepackage{tikz}
\usepackage{fancyvrb}

\usepackage{etoolbox}
\patchcmd{\section}{\scshape}{\Large\scshape}{}{}

\usepackage[width=6.5in,height=9.5in]{geometry}


% colored PDF links
\usepackage[bookmarks,colorlinks,breaklinks]{hyperref} 
\usepackage{color}
\definecolor{dullmagenta}{rgb}{0.4,0,0.4}   % #660066
\definecolor{darkblue}{rgb}{0,0,0.4}
\hypersetup{linkcolor=red,citecolor=blue,filecolor=dullmagenta,urlcolor=darkblue} % coloured links

% header and footer 
\usepackage{fancyhdr}
\usepackage{lastpage}
\pagestyle{fancy}
\fancyhf{} % Removes everything
\lhead{Team \# 54328}
\rhead{Page \thepage\ of \pageref*{LastPage}}

\DeclareMathOperator*{\argmax}{arg\,max}

\setlength{\skip\footins}{2cm}

\renewcommand{\Re}{\mathbbm{R}}
\newcommand{\UU}{\mathcal{U}}
\newcommand{\D}{\mathcal{D}}
\newcommand{\V}{\mathcal{V}}
\newcommand{\T}{\mathcal{T}}
\newcommand{\W}{\mathcal{W}}
\newcommand{\dV}{\frac{d\mathcal{V}}{dt}}


\makeatletter
\renewcommand{\@maketitle}{
	\newpage
	\null
	\vskip 2em%
	\begin{center}%
		{\LARGE \@title \par}%
	\end{center}%
	\par\vspace{2em}} \makeatother


\title{[TITLE]}

\begin{document}
	\nocite{*}
	\thispagestyle{empty}
	\ThisULCornerWallPaper{1}{54328sum.pdf}
	% Make sure you replace this .pdf file with *your* summary sheet 
	
	\ \vspace{8.6cm}
	
		This paper outlines a model that ranks universities in the U.S. according to their potential to increase student success on their campuses. We incorporate machine learning in our model (SVM regressor, recurrent neural networks) and cluster analysis to to achieve a ranked university list, an analysis of utility functions pulled from the literature to determine the amount of grant money allotted to each ranked university, and mathematical analysis to derive the time over which each grant will be distributed. These three components make up our investment plan for the Foundation.
		
		The model seeks to identify the universities that could make the greatest impact after being supplied with funds from the Foundation. In order to quantify this potential impact, we combine three sets of data: the two supplied sets in addition to university financial data collected by The Delta Project. Each of these sources collected various pieces of information about each school; we refer to these pieces of information as variables. By tracking how these variables change over time, we can predict how each school will use its finance variables to affect its institutional variables. In traditional grand-awarding procedures, each of these variables is given importance, or weight, by a panel of application reviewers, i.e. humans. Our model avoids this human panel by applying weight to each variable according to a clustering algorithm. The algorithm reviews a data set and groups (clusters) variables that are similar (be it in number, separation, etc.). This program then learns the difference between ``good'' variables, or attributes that make a university a good candidate for the grant, and ``bad'' variables. The good and bad variables are given a weight based on their calculated ``goodness''. We aggregate the weighted variables and optimize the change that occurs between them, thereby optimizing the potential that any university may have to propagate success with their grant money. This is how our universities are ranked.
		     
		The amount given to each school is determined by an analysis of utility functions. We find the utility of giving money to a highly-ranked school increases according to a modified logarithmic function. As utility increases, so does the amount of grant money given to a certain school. Utility is largely dependent the score given to each university during the ranking process. 
		     
		The model's time line for distribution of funds over a five year period is simple: we expect that all funds will be allocated as soon as ranks are released. This assumption comes from data generated in an earlier part of the model. Following several assumptions, our model should acceptably meet the demands of the Goodgrant Foundation.
	
	\clearpage
	\setcounter{page}{1}
	\maketitle
	%\tableofcontents
	%\pagebreak
		
	\thispagestyle{empty}
	
%%%%%%%%%%%%%%%%%%% Now, we just include each of the sections. %%%%%%%%%%%%%%%%%%
	\subfile{intro}
	\subfile{model}

	\subfile{work_appendix}
	\subfile{ml}
	\subfile{cluster}
	\subfile{funds}
	\subfile{timeline.tex}
	\subfile{discussion}

	\bibliographystyle{acm}
	\bibliography{references}
	
\end{document}
