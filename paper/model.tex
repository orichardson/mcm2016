% !TeX root = model.tex

\documentclass[paper.tex]{subfiles}

\newcommand{\U}{\mathcal{U}}
\newcommand{\D}{\mathcal{D}}
\newcommand{\V}{\mathcal{V}}
\newcommand{\T}{\mathcal{T}}
\newcommand{\W}{\mathcal{W}}
	

\begin{document}

%	\section{The Model}
%	The model utilizes two data sets throughout the project. The first set of data was collected by The Delta Project \cite{desrochers2010trends} regarding university finances. The second set comes from CollegeScoreCard and NCES. From the data we create a n-dimensional space that represents all the money a university has at its disposal from third-party donors, such as the Goodgrant Foundation. The data are also associated with certain characteristics about each university, such as  	The first part of the model generates a function that maps each university's disposable donations to 
	
	\section{Formal Prolegomena}
	
	First, some notational definitions. We have tabulated them below. 
	
	
	\begin{itemize}
		\item[($\U$)] The set of universities and colleges in question.
		\item[($\D$)] Financial space. In general, this is the space of donations -- this might have multiple dimensions over $\Re$, depending on the specific categories of money we're interested in. It might also contain certain categories of expenditures that we might be able to influence by placing constraints on how institutions are allowed to spend their money.  
		\item[($\T$)] The space of times for which we have data. 
		\item[($\V$)] The vector space of student metric variables $\{v_i\}$. Note that at this point, we have not yet committed ourselves to any such choice of variables, and so $\V$ includes also negative and neutral indicators of success.
		\item[($d\V)$] In a continuous setting, this would be the $\V$ differential 1-form, but here it is simply a running difference of the variables $\V$ over time.  
	\end{itemize}
			
	We will also be interested in a ``sliding window'' of times trailing a given time; if we're interested in a subspace of time $T \subset \T$, we can construct this window as $\W_T : T  \to \D$. Note that in the case of a discrete $T$, comprised of $n$ time steps, $\W_T = \D^n$. All of our data is discrete, of course, but this construction also works for continuous subspaces of $\T$, which means we can think of different choices of $T$ as appropriate approximations, instead of sets with fundamentally different structures. It is also useful to include a shorthand several particularly important parameters
	
	\begin{itemize}[leftmargin=5em]
		\item[($\mathcal{M} \subset \D$)] The space of donations. This is a special subset of $\D$, as we are constrained by a total sum of money, and trying specifically to optimize with respect to this quantity.
		\item[($M_u \in \mathcal{M}$)] The total amount of money given to university $u$ over the entire time $T$
	\end{itemize}
	
	With this framework, we can now formulate the problem more precisely. To do any kind of induction at all, it is necessary to make some commonplace but sometimes very wrong independence assumptions. Here's our first and most central one: we will assume that the effectiveness with which an institution can use money does not change over time\footnote{This is a reasonable assumption to make; while technically invalid, it seems very natural to judge an institution by its past performance -- indeed, this is the best we can hope for from a dataset}. For an institution $u \in \U$, we can now talk about the effect of donor money over time on the variables in $\V$ as a mapping
	\begin{equation}
		F_u: \W_T \to d\V \label{F}
	\end{equation} 	
	That is -- a function which takes the financial input we control, and changes student metric variables in some way. This is to be distinguished from some other function $\phi_u: \W_T \to \V$ that estimates the metric variables directly. There are several reasons for this. First of all, it makes a lot more semantic sense; we can't directly impact the absolute value of the these variables with money; at best, we nudge them in one direction or another. Secondly, training on the raw values incentivizes memorizing institutions' particular statistics instead of learning a general pattern. Speaking of which, to get this kind of general pattern instead of one for each school, we are actually learning an alternate formulation of (\ref{F}) -- 
	\[F: \W_T \times \U \to d\V \]
	
	Supposing that we have a suitably robust approximation for $F$, we still have a ways to go in terms of allocating resources. The biggest, most glaring issue is that we still haven't decided what makes a given change $d\V$ a \emph{positive} one. The mechanism for determining this will be discussed at length later in the paper, but here we will just give it a name. Let
	\begin{equation}
		A : \V \to \Re
	\end{equation}
	be an aggregation function, which produces an objective ``goodness'' of a given array of metric variables. Note that here we use $\V$, not $d\V$, even after making such a point of doing precisely the opposite with respect to $F$. We do this because, \emph{a priori}, it is possible that schools have intrinsic properties that impact the effectiveness of donations, but which do not actually themselves change with respect donation amounts. This is pretty inconvenient, because we really wanted to compose $A$ with $F$ to get a complete function with which to score elements of $\D$. For a moment, we'll suppose that we had the hypothetical function $\phi$ described above. In that case, we have
	\[ \W_T \overset{\phi}{\to} \V \overset{A}{\to} \Re \] 
	\ldots but this isn't exactly the metric we're after. Instead of measuring the effectiveness of a contribution, we have some absolute measure of how ``good'' a school would be with a given donation distribution.
	\begin{equation}
		\argmax_{u \in \U}~ \frac{d}{dM_u} A\left(V_u + \int_{T} F(w_T, u)\right)
	\end{equation}
	where $V_u \in \V$ is the most current metric vector for university $u$.  
	\begin{align*}
		&=\argmax_{u \in \U}~ \frac{d}{dM_u} A\left(V_u + \int_{T} F(w_T, u)\right) \\
		&=\argmax_{u \in \U}~  A'\left(V_u + \int_{T} F(w_T, u)\right)\frac{d}{dM_u}\int_{T} F(w_T, u) \\
		&=\argmax_{u \in \U}~  A'\left(V_u + \int_{T} F(w_T, u)\right)\int_{T} \frac{\partial F}{\partial M_u}(w_T, u) \\
		&=\argmax_{u \in \U}~  A'\left(V_u + \int_{T} F(w_T, u)\right)\int_{T} F'(w_T, u) \frac{\partial w_T}{\partial M_u} \\
	\end{align*}
	\begin{enumerate}
		\item Apply Machine Learning techniques to estimate the effect of donations on the 
	\end{enumerate}
\end{document}